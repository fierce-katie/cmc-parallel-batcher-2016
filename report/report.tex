\documentclass[oneside,final,14pt]{extreport}
\usepackage[T1,T2A]{fontenc}
\usepackage[utf8]{inputenc}
\usepackage[russian]{babel}
\usepackage{vmargin}
\usepackage{textcomp}
\newcommand*{\No}{\textnumero}
\setpapersize{A4}
\setmarginsrb{3cm}{2cm}{1cm}{2cm}{0pt}{0mm}{0pt}{13mm}
\usepackage{indentfirst}
\sloppy
\setcounter{page}{2}
\clubpenalty = 10000
\widowpenalty = 10000
\setlength{\parskip}{0.3cm}

\begin{document}

\tableofcontents

\chapter*{Постановка задачи}
\addcontentsline{toc}{chapter}{Постановка задачи}

В задаче требовалось реализовать параллельную сортировку Бэтчера для структур,
представляющих точки на регулярной сетке (\texttt{Point}), вдоль одной из
координат ($x$ или $y$).

Структура \texttt{Point} имеет следующий вид:
\begin{verbatim}
    Point {
        float coord[2];
        int index;
    };
\end{verbatim}

Пусть сетка размерности $n_1 * n_2$ представляется массивом таких структур
\texttt{P[n1 * n2]}, а для инициализации координат точек используются функции
\texttt{float x(int i, int j)} и \texttt{float y(int i, int j)}.
Тогда \texttt{P[i*n2+j].coord[0] = x(i, j)},
\texttt{P[i*n2+j].coord[1] = y(i,j)}, \texttt{P[i*n2+j].index = i*n2+j}, где $i = \overline{0, n_1-1},
j = \overline{0,n_2-1}$.

На каждом процессоре должно обрабатываться одинаковое количество элементов.
Каждый процессор выполняет упорядочивание элементов независимо от других.
Слияние каждого отсортированного массива должно происходить в соответствии с
расписанием, задаваемым сетью сортировки Бэтчера.

После окончания работы программы на каждом процессе должно находиться одинаковое
количество элементов структуры \texttt{Point}. Каждый элемент структуры
\texttt{Point} одного процесса должен находиться левее по координате
по сравнению с элементом структуры \texttt{Point} любого другого процесса с
б\'{о}льшим рангом.

Программа должна демонстрировать эффективность не менее 80\% от максимально
возможной на числе вычислительных ядер не менее 128.

\chapter*{Метод решения}
\addcontentsline{toc}{chapter}{Метод решения}

\section*{Построение сети слияния}
\addcontentsline{toc}{section}{Построение сети слияния}

\section*{Распределение элементов по процессорам}
\addcontentsline{toc}{section}{Распределение элементов по процессорам}

\section*{Сортировка элементов}
\addcontentsline{toc}{section}{Сортировка элементов}

\section*{Слияние отсортированных фрагментов}
\addcontentsline{toc}{section}{Слияние отсортированных фрагментов}

\chapter*{Используемая вычислительная система}
\addcontentsline{toc}{chapter}{Используемая вычислительная система}

\chapter*{Анализ полученных результатов}
\addcontentsline{toc}{chapter}{Анализ полученных результатов}

\end{document}


